\section{Test}
This section contains all the test-cases we have deviced to show that the program can successfully complete all the requirements. 

\subsection*{Test 1: Creation of group}
\paragraph{Purpose:} Test to show that the group module can create groups and that the name server will store them with the correctl leader. 
\begin{table}[h!]
	\begin{tabularx}{\textwidth}{|l|X|p{8pt}|}
		\hline
		1 & Start the name server. & \\
		\hline
		2 & Start client 1. & \\
		\hline
		3 & Create a group with client 1. &  \\
		\hline
		4 & Start client 2. & \\
		\hline
		5 & The group should exist with client 1 as leader in client 2. & \\
		\hline
	\end{tabularx}
\end{table}

\paragraph{Expected result:} The group exists on the name server with Client 1 as leader.

\subsection*{Test 2: Member joining \& basic multi-cast}
\paragraph{Purpose: } Test to show that a client can join a newly created group and send a message.
\begin{table}[h!]
	\begin{tabularx}{\textwidth}{|l|X|p{8pt}|}
		\hline
		1 & Start the name server. & \\
		\hline
		2 & Start client   1. & \\
		\hline
		3 & Create a group with client 1. & \\
		\hline
		4 & Start client   2. & \\
		\hline
		5 & Join existing group with client 2. & \\
		\hline
		6 & Send a message from either client. & \\
		\hline
		7 & Both clients should receive the message.  & \\
		\hline		
	\end{tabularx}
\end{table}
\paragraph{Expected result:} Both clients display the message.

\subsection*{Test 3: Member leaving}
\paragraph{Purpose:} Test to show that a member can leave/crash and the other clients will detect and not propagate messages to them.
\begin{table}[h!]
	\begin{tabularx}{\textwidth}{|l|X|p{8pt}|}
		\hline
		1 & Start the name server. & \\
		\hline
		2 & Start client   1. & \\
		\hline
		3 & Create a group with client from 2. & \\
		\hline
		4 & Start client   2. & \\
		\hline
		5 & Join existing group with client 2. & \\
		\hline
		6 & Start client   3. & \\
		\hline
		7 & Join existing group with client 3. & \\
		\hline
		8 & Send a message from client 3. & \\
		\hline		
		9 & Leave group with client 3. & \\
		\hline
		10 & Client 1 or 2 sends a message and sees that it doesn't send to client 3 and doesn't crash.& \\
		\hline
		11 & Client 1 and 2 should display the received message.& \\
		\hline		
	\end{tabularx}
\end{table}

\paragraph{Expected result: } Client 1 and 2 displays the message without crashing.

\subsection*{Test 4: Causal, reordering \& hold-back queue}
\paragraph{Purpose: } Test to show that the debug client can re-order messages and display the hold-back queue in order to also show that causal ordering works correctly.

\begin{table}[h!]
	\begin{tabularx}{\textwidth}{|l|X|p{8pt}|}
		\hline
		1 & Start the name server. & \\
		\hline
		2 & Start client 1 with causal message ordering. & \\
		\hline
		3 & Create a group with client 1. & \\
		\hline
		4 & Start client 2 with causal message ordering. & \\
		\hline
		5 & Join existing group with client 2.& \\
		\hline
		6 & Client 2 activates reordering. & \\
		\hline
		7 & Client 1 sends $n$ messages where $n$ is enough messages to put some messages in hold-back queue. & \\
		\hline
		8 & Client 2 deactivates reordering. & \\		
		\hline
		9 & Client 1 sends another message. & \\
		\hline
		10 & Both clients should display all the messages in the correct order. & \\
		\hline
	\end{tabularx}
\end{table}

\paragraph{Expected result:} All messages are displayed in the correct order in both clients.

\subsection*{Test 5: Member joining causal ordering group in progress}
\paragraph{Purpose: } Test to show that a client can join a group that has already sent messages with causal ordering,
to show that a client can inherit the logical clock when joining a group.
\begin{table}[h!]
	\begin{tabularx}{\textwidth}{|l|X|p{8pt}|}
		\hline
		1 & Start the name server & \\
		\hline
		2 & Start client 1 with causal message ordering. & \\
		\hline
		3 & Create a group with client 1. & \\
		\hline
		4 & Start client   2 with causal message ordering. & \\
		\hline
		5 & Join existing group with client 2.& \\
		\hline
		6 & Send a message from either client 1 or 2. & \\
		\hline
		7 & Start client   3 with causal message ordering. & \\
		\hline		
		8 & Join group with client 3. & \\
		\hline		
		9 & Send message from either client 1 or 2. & \\
		\hline		
		10 & Client 3 should receive the message. & \\
		\hline		
	\end{tabularx}
\end{table}

\paragraph{Expected result:} Client 3 displays the message.


\subsection*{Test 6: Unordered message ordering}
\paragraph{Purpose: } Test to show that clients can use unordered message ordering.
\begin{table}[h!]
	\begin{tabularx}{\textwidth}{|l|X|p{8pt}|}
		\hline
		1 & Start the name server & \\
		\hline
		2 & Start client 1 with unordered message ordering. & \\
		\hline
		3 & Create a group with client 1. & \\
		\hline
		4 & Start client 2 with unordered message ordering. & \\
		\hline
		5 & Join existing group with client 2.& \\
		\hline
		6 & Enable reordering in client 2. & \\
		\hline
		7 & Send messages from client 1. & \\
		\hline		
		8 & Messages in client 2 should arrive unordered.&\\
		\hline		
	\end{tabularx}
\end{table}

\paragraph{Expected result:} Client 2 displayes the messages unordered.
\pagebreak

\subsection*{Test 7: Inheriting message ordering}
\paragraph{Purpose: } Test to show that a client starting a group with unordered message ordering will cause any joining client to use unordered message ordering.
\begin{table}[h!]
	\begin{tabularx}{\textwidth}{|l|X|p{8pt}|}
		\hline
		1 & Start the name server & \\
		\hline
		2 & Start client 1 with unordered message ordering. & \\
		\hline
		3 & Create a group with client 1. & \\
		\hline
		4 & Start client 2 with causal message ordering. & \\
		\hline
		5 & Join existing group with client 2.& \\
		\hline
		6 & Enable reordering in client 2. & \\
		\hline
		7 & Send messages from client 1. & \\
		\hline		
		8 & Messages in client 2 should arrive unordered. &\\
		\hline		
	\end{tabularx}
\end{table}

\paragraph{Expected result: } Client 2 displayes the messages unordered.

\subsection*{Test 8: Leader leaving and other process takes over leader}
\paragraph{Purpose: } Test to show that if the leader leaves/crashes another process will take over leadership and processes can rejoin

\begin{table}[h!]
	\begin{tabularx}{\textwidth}{|l|X|p{8pt}|}
		\hline
		1 & Start the name server. & \\
		\hline
		2 & Start client 1.  & \\
		\hline
		3 & Create a group with client 1. & \\
		\hline
		4 & Start client 2. & \\
		\hline
		5 & Join existing group with client 2 (Note that leader is client 1).& \\
		\hline
		6 & Close down client 1. & \\
		\hline
		7 & Start client 3. & \\
		\hline
		8 & Group list for client 3 should now show that client 2 is leader & \\
		\hline
	\end{tabularx}
\end{table}

\paragraph{Expected result: } Group list for last started client should show client 1.
