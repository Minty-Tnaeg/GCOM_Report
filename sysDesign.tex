\section{System Design}
This section of the report contains the explanation of how and why the system was implemented as it is.
The client and name server have simple designs and will hence not be touched in this section. The \mw\ is more complex and will construct the major part of this section.

\subsection{Module overview}
The \mw\ is constructed based around three modules. A group management module, a message ordering module and a communication module.
Some models exists for such a module system for example where each module is placed on top of each other. Where communication is done from the communication module up to the message ordering module then to the group module and finally presented to the client (see figure \ref{fig:classic}).
\begin{figure}[h!]
	\centering	
\begin{tikzpicture}[ 
	font=\sffamily,
	every matrix/.style={ampersand replacement=\&,column sep=0.5cm,row sep=0.5cm},	
	server/.style={rectangle, draw, fill=red!40, minimum height=1cm, minimum width=2cm, font=\ttfamily\footnotesize},
	user/.style={rectangle, draw, fill=blue!40, minimum height=1cm, minimum width=2cm, font=\ttfamily\footnotesize},
	back/.style={rectangle, draw, fill=orange!50, minimum height=1cm, minimum width=2cm, font=\ttfamily\footnotesize},
	com/.style={rectangle, draw, fill=green!50, minimum height=1cm, minimum width=2cm, font=\ttfamily\footnotesize},
	both/.style={<->,>=stealth',shorten >=1pt,semithick,font=\sffamily\footnotesize},	
	to/.style={->,>=stealth',shorten >=1pt,semithick,font=\sffamily\footnotesize},
	every node/.style={align=center}]  
\matrix (m){
	\& \node[server] (client) {Client}; \\
	\& \node[user] (group) {Group Module};  \\
	\& \node[back] (message) {Message ordering}; \\ 
    \& \node [com] (com) {Communication};\\
	};	
	\draw[both] (client) -- (group);
	\draw[both] (group) -- (message);
	\draw[both] (message) -- (com);
	\draw[to] (com) -- (0.3,-3.5);
	\end{tikzpicture}
	\caption{Group communication overview}
	\label{fig:classic}
\end{figure}

The implementation for this project does however not follow that, instead it places the modules around a central module. This is done because it allows better for parallel implementation and reduces dependency between modules (see figure \ref{fig:imp}).
\begin{figure}[h!]
	\centering	
\begin{tikzpicture}[ 
	font=\sffamily,
	every matrix/.style={ampersand replacement=\&,column sep=0.5cm,row sep=0.5cm},	
	client/.style={rectangle, draw, fill=white!40, minimum height=1cm, minimum width=2cm, font=\ttfamily\footnotesize},
	server/.style={rectangle, draw, fill=red!40, minimum height=1cm, minimum width=2cm, font=\ttfamily\footnotesize},
	user/.style={rectangle, draw, fill=blue!40, minimum height=1cm, minimum width=2cm, font=\ttfamily\footnotesize},
	back/.style={rectangle, draw, fill=orange!50, minimum height=1cm, minimum width=2cm, font=\ttfamily\footnotesize},
	com/.style={rectangle, draw, fill=green!50, minimum height=1cm, minimum width=2cm, font=\ttfamily\footnotesize},
	both/.style={<->,>=stealth',shorten >=1pt,semithick,font=\sffamily\footnotesize},	
	to/.style={->,>=stealth',shorten >=1pt,semithick,font=\sffamily\footnotesize},
	every node/.style={align=center}]  
\matrix (m){
	\& \node[client] (client) {Client}; \& \\
	\node[back] (message) {Message Module}; \&
	\node[server] (middleware) {Middleware}; 
	\& \node[user] (group) {Group Module}; \\
	\node[com] (com) {Communication Module}; \\
	};	
	\draw[both] (client) -- (middleware);
	\draw[both] (middleware) -- (group);
	\draw[both] (middleware) -- (message);
	\draw[both] (com) -- (message);
	\end{tikzpicture}
	\caption{Group communication overview}
	\label{fig:imp}
\end{figure}

\subsubsection{Group Module}
In the implementation described in this document the group module has the tasks of handling all group management issues presented in section \ref{sec:group}. That being methods that allow processes to join, create, leave and list members of a group. This module keeps track of all network proxys that are given by other processes. The group module is represented in blue in figure \ref{fig:imp} above.

\subsubsection{Message Module}
The message module that is represented in orange in figure \ref{fig:imp} above. The module keeps handles the issues presented above in section \ref{sec:messageOrdering}. Most of this is done with the help of abstract containers for the orderings. Where the containers wrap messages depending on a chosen type of container. The container types available is the \texttt{Causal} container and a \texttt{Unordered} container. The \texttt{Causal} container contains methods for comparing the logical clocks and then decides if the message should be presented or held back.

The module parses two main types of messages, system messages and text messages. System messages are the following:

\begin{itemize}
	\item Join: Indicating a process wishes the to join the receivers group.
	\item Election: The leader has crashed or left, an election has started.
	\item Leave: A user has left the group. Message contains the user who left.
	\item Error: Something has gone wrong for a process (a connection has failed).
	\item New leader: A new leader was elected. Message contains leader.
	\item Return election: Answer to election message to indicate alive.
	\item Return Join: Gives the joining process every available process in the group.
\end{itemize}

The text message is just a container for the text sent by a process.

\subsubsection{Communication module}
This is the module that is represented in green in figure \ref{fig:imp}. The module implements basic multi-cast. 

 




